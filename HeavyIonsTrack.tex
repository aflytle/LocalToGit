\documentclass{article}
\usepackage{mathtools}
\usepackage{tikz}

\begin{document}
\title{Raw Notes from Group Meeting 24/May/2019}
\author{Aidan Lytle}

\maketitle

\section{Natural Units}
\vspace{10 mm}

Natural Units: fundamental constants are all set to unity.\\
Unit of quantum action: $\hbar$ = 1. \\
Speed of light: $c$ = 1. \\
Boltzmann constant: $k_B$ = 1. \\
(What we do is thermal shtuff, some concept of temperature.)

\section{Typical Units}

\begin{itemize}
\setlength\itemsep{0em}
\item Mass of proton = 938.3 MeV = 1.007 amu = $1.673 \times 10^{-27}$ kg
\item Typical Energy = 1 GeV = $1.602 \times 10^{-10}$ J
\item Typical size = 1 fm = $10^{-15}$ m
\item Typical time = 1 fm = $3.336 \times 10^{-24}$ s
\item Typical Temperature = 200 MeV = $2.321 \times 10^{12}$ K
\newline (we call really hot astrophysical objects temperature zero)
\newline (Temperature is derivative of energy with respect to entropy)
\end{itemize}

\section{History of Concept}
\begin{itemize}
\setlength\itemsep{0em}
        \item Quantum Chromodynamics: Theory of the strong interaction
        \item Physics begins in 400 BC with Democritus' theory of atoms.
        \item 1687 - newton's philosophiae
        \item 1900 - Planck's law of blackbody radiation and entropy
        \item 1905 - Einstein's four papers
        \item 1911 - Rutherford Scattering
        \item 1913 - Bohr atom
        \item 1924 - De Broglie's matter waves and wavelengths
        \item 1925 - Heisenberg's Matrix mechanics
        \item 1926 - Schroedinger equation
        \item 1927 - Dirac's relativistic Quantum Mechanics
        \item 1963 - Gell-Mann's Quark Model
        \item 1965 - Additional Quark degree of freedom (colour)
        \item 1969 - Deep inelastic Scattering experiments prove existence of Quarks
        \item 1972 - Color sharge and basic frameworks of QCD
        \item 1973 - Asymptotic freedom discovered by Gross, Politzer, Wilczek $\rightarrow$ when energy is arbitrarily high,
             strength of interaction of color is zero.
        \item 1975 - Collins and Perry formulate a QCD plasma (QGP) $\rightarrow$ this is based on 'deconfinement', or the theory of
             the deconfinement of quarks at ultra-high energy into a charged plasma
        \item 1980 - Shuryak coins QGP as a term
        \item 2000 - RHIC is operational
        \item 2010 - First Heavy Ion collisions at LHC
\end{itemize}

\section{Concepts of QCD and General Field Theories}

\begin{itemize}
\setlength\itemsep{0em}
\item QCD is theory of strong interactions
\item Quarks and gluons are fundamental particles
\item There are three colors and eight kinds of gluon
\item QCD exhibits confinement meaning only found in bound states
\item QGP $\rightarrow$ quarks interacting with each other gain a bunch of dynamical mass that is outside their Higgs Mass
\item 99\% of the mass of the visible universe comes from QCD
\item QFT is only consistent way to combine Special Relativity and Quantum Mechanics $\rightarrow$ look up Steven Weinberg's proof
\item Baryons and Mesons Color-Charged paricles (quarks and gluons) are called partons
\item QCD bound states are called hadrons
\item All observables must be in color singlet state - no partons can be found in isolation in nature
\item Charge conjugate gives us the antiparticle
\end{itemize}

\newpage

\noindent The QED potential is given by
\begin{equation*}
V(r) = -\frac{\alpha_{EM}}{r},
\end{equation*}
where $\alpha_{EM}$ is the QED coupling (aka fine structure constant), given by
\begin{equation*}
\alpha_{EM} = \frac{e^2}{4\pi}.
\end{equation*}


\noindent The QCD potential for $q\bar{q}$
\begin{equation*}
V(r) = -\frac{4}{3}\frac{\alpha_S}{r} + kr,
\end{equation*}
where 4/3 is the fundamental casimir in the QCD gauge group $SU(3)$
and $\alpha_S$ is the QCD coupling, given by
\begin{equation*}
\alpha_s = \frac{g^2}{4\pi}.
\end{equation*}


\noindent New pairs are created when energy exceed mass (when the string tension between two quarks is greater than the quark mass)
a new quark antiquark pair is generated

\noindent Some quark contents of common hadrons:
$p = uud$, $n = udd$, $\pi^+ = u\bar{d}$, $\pi^- = d\bar{u}$, $\pi^0 = (u\bar{u} - d\bar{d})/\sqrt{2}$
%Pi plus meson, pi minus meson, pi zero - up anti-up, down anti-down

\end{document}
